\chapter{Discussion and Conclusions}
\label{sec:discuss}

In this work, we have examined the feasibility of, and the computational benefit afforded by utilizing Adaptive Mesh Refinement (AMR) to adaptively refine simulations of accretion disks during the linear growth phase of the magnetorotational instability (MRI). The criterion we have used is \citet{2004ApJ...605..321S}'s resolution criteria for the resolution of MHD turbulence driven by the MRI, - $\langle \langle \lambda^2_\text{MRI}\rangle\rangle ^{1/2} \gtrsim 6 \Delta$, as generalized to the resolution of the linear MRI locally. This defines the well-studied quality factor metric for analyzing convergence of MHD simulations of disks.

Through the 2D axisymmetric simulations performed in this study, we find that AMR simulations utilizing this criterion show agreement in both space-integrated global quantities like the total mass $M$ and the poloidal magnetic energy $B_\text{pol}$, and spatially varying time and z-averaged radial quantities (in particular $\alpha_\text{mag}$). Because AMR simulations refine the computational domain selectively, the total number of cells is lower for the AMR simulations and we observe a factor $\sim 2-3$ reduction in the total number of cell advances for the AMR simulations as compared to the comparable uniform grid simulations.

Cowling's antidynamo theorem \citep{1933MNRAS..94...39C} means that our simulations cannot sustain a poloidal magnetic field owing to the two-dimensional nature of our simulations. Analyzing the applicability of this resolution criteria to 3D non-axisymmetric simulations is one of the avenues for future work. In a recent study, \citet{jannaud_numerical_2025} analyzed the issues with utilizing the quality factor as a metric of convergence and as a resolution criteria, highlighting issues particularly in the case of a zero net magnetic flux. They also highlight issues with the assumptions underpinning the quality factor - a linear theory assuming a constant net magnetic flux and thus, a well-defined minimum MRI lengthscale. They instead propose a revised quality factor, which is a function of the net magnetic flux threading the disk and the quality factor associated with said flux.

Our work offers a contrast to that study, showcasing the applicability of the quality factor discussed in literature to at least 2D axisymmetric simulations of accretion disks during the linear growth phase of the poloidal MRI. At the very least, we hope to inspire future work utilizing Adaptive Mesh Refinement (AMR) as a tool to better utilize available computing resources in MHD simulations of disks.