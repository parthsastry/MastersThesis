\chapter{Introduction}
\label{sec:intro}

\pagenumbering{arabic} 

Astrophysical systems are inherently multi-scale, with physically relevant processes often occurring at length scales that span many orders of magnitude. This is readily apparent in astrophysical plasmas in a variety of contexts - the interstellar medium (ISM), protoplanetary and accretion disks, the solar corona, and molecular clouds. With the rise of computing capabilities, numerical simulations have become an important tool in understanding the evolution of these systems. Capturing the essential dynamics of the system necessarily entails that simulations incorporate physical processes either by fully resolving the requisite length scales in a direct numerical simulation approach, or otherwise capturing them via subgrid models. In this thesis, we will focus specifically upon direct numerical simulations. 

%Due to the inherently multi-scale nature of such systems, 
Dynamically relevant length scales will generally exhibit spatial variation over the simulation domain as well as temporal variation as the simulation evolves. A conservative approach to capturing these relevant length scales involves uniformly resolving the smallest length scale across the entire simulation. However, this approach is generally computationally expensive, with some regions of the simulation domain often over-resolved at some time steps. For example, in the context of self-gravitational hydrodynamics, it was found that numerical instabilities can be avoided by ensuring that the Jeans length is always resolved \citep{1997ApJ...489L.179T}. %number $J \equiv \Delta x/\lambda_J \leq 0.25$, called the Jeans condition. Here, $\lambda_J = ( \pi c_s^2 / G \rho )^{1/2}$ is the Jeans length (\cite{1902RSPTA.199....1J}, \cite{Jeans1928-JEAAAC}).
Because the Jeans length is not a constant across space and time, simulations that locally resolve the Jeans criterion using adaptive mesh refinement, smoothed particle hydrodynamics, or a moving mesh are vastly less computationally expensive than maximally refining the Jeans length everywhere and all times, while also ensuring numerical stability and convergence \citep{hummels_adaptive_2012}.
% This text is redundant
%Adaptive Mesh Refinement (AMR) simulations offer an alternative to this by adaptively refining the grid based upon a relevant criterion.

This paradigm of tracking characteristic length scales is applicable to a wide range of physical processes, and has been applied for example to thermal conduction instabilities \citep{koyamainutsuka04, gressel09}, magnetized self-gravitating turbulent dynamos \citep{federrathetal11}, and gaseous cooling \citep{reyetal24}.
%It was shown in \cite{1965ApJ...142..531F} that there exists a stability criterion for a dilute gas in thermal and mechanical equilibrium and that a near-isentropic sound wave can grow in amplitude if an instability condition is satisfied, causing condensations of density. If this is the relevant dynamical phenomenon in the system, then tracking the fastest growing unstable mode of this instability via AMR simulations should suffice to capture essential dynamics of the system.
In this thesis, we explore length scale tracking in the context of the magnetorotational instability (MRI, \cite{1991ApJ...376..214B}, \cite{RevModPhys.70.1}) in accretion disks, where the MRI is generally understood to be a primary driver of disk turbulence and disk evolution, causing radially outward angular momentum transfer and inward mass accretion through the action of the combined Reynolds and Maxwell turbulent stresses.

Because of its key role in driving angular momentum transport, it has long been understood that resolving the fastest-growing MRI mode $\lambda_{\text{MRI}}$ is important to accurately capture accretion disk dynamics in MHD simulations. For example, through shearing box simulations, \citet{2004ApJ...605..321S} found that to resolve MHD turbulence driven by the MRI, one needs to have $\langle \langle \lambda_{\text{MRI}}^2 \rangle \rangle^{1/2} \gtrsim 6 \Delta$, where $\Delta$ is the grid length, and the double brackets indicate a volume and time average. Simulations that meet this criteria accurately capture the linear growth of the fastest-growing mode as well as the saturation limits of the magnetic field.

In an axisymmetric geometry, the MRI wavelength $\lambda_{\text{MRI}}$ is determined from the vertical magnetic field component $B_z$ \citep{1991ApJ...376..214B}, i.e.,

\begin{equation} \label{eqn:lambdamri}
     \langle \lambda_{\text{MRI}}^2 \rangle^{1/2} = 2\pi \frac{\langle v_{\text{A}z}^2 \rangle^{1/2}}{\Omega} = \frac{2\pi}{\Omega} \left( \frac{\langle B_{z}^2 \rangle }{4 \pi \rho} \right)^{1/2}
\end{equation}
%
Here $v_{\text{A}z}$ is the Alfv\'en velocity derived from the vertical magnetic field, $\Omega$ is the local disk angular velocity, and $\rho$ is the mass density.


An initial seed magnetic field undergoes exponential growth via the MRI, eventually saturating when the fastest growing mode becomes of order the disk scale height, $\lambda_{\text{MRI}} \sim h \sim \Omega / c_s$. Frequently, the seed magnetic field for the growth of the MRI is orders of magnitude weaker than its final saturation strength, implying that one needs to capture a dynamic range $h / \lambda_{\text{MRI}, 0} \sim 2 \pi c_s / v_{A,0}$, where $\lambda_{\text{MRI}, 0}$ and $v_{A,0}$ are the fastest growing mode length and Alfv\'en speed based upon the initial seed magnetic field. For example, \citet{kiuchi_large-scale_2024} found that in binary neutron star mergers, an initial seed field of $\sim 10^{44}$ erg at time of merger increases by roughly 4 orders of magnitude to $\sim 10^{48}$ erg within roughly 2 ms after merger.

Consequently, a key challenge to numerical simulations is to be able to capture both the growth of the magnetic field and the development of disk turbulence under the action of the MRI, while also simultaneously resolving the global disk structure. One must either use extremely high resolution to capture the fastest growing MRI mode of the weak seed magnetic field, or otherwise utilize an artificially strong magnetic field.  

In Section \ref{sec:simulation}, we present our  proposed AMR criterion for MRI disks, as well as the numerical simulations for verification. We also discuss the effect of simulation dimensionality on the proposed criterion and its extension to more general $3$D simulations. In Section \ref{sec:results}, we compare simulations utilizing the proposed AMR criterion and reference uniform grid simulations via a variety of diagnostics to qualitatively and quantitatively assess closeness of simulations. In Section \ref{sec:discuss}, we discuss some of the implications of these results, follow-up on our note on simulation dimensionality with some discussion of the extension of the AMR criterion to $3$D and conclude our work. 