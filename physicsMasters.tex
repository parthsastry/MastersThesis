\documentclass[12pt,defaultstyle]{umassdthesis}

\usepackage[T1]{fontenc}
\usepackage[latin1]{inputenc}
\usepackage{graphicx, subcaption}
\usepackage{float}
\usepackage{hyperref}
\hypersetup{
colorlinks=true,
linkcolor=,
citecolor=blue
}
\urlstyle{same}
\usepackage{multirow,hhline,dcolumn, booktabs} % useful commands for Tables
\usepackage{natbib}
\usepackage{amsmath, amssymb}
%\usepackage[toc]{appendix}
%\usepackage[pages]{appendix}

%%  user defined commands
\newcommand{\linefrac}[2]{\raisebox{.6ex}{#1}/\raisebox{-0.6ex}{#2}} 
\def\BibTeX{{\rm B\kern-.05em{\sc i\kern-.025em b}\kern-.08emT\kern-.1667em\lower.7ex\hbox{E}\kern-.125emX}}

%%  the entries below should be self-explanatory, some of these entries may
%%  safely be excluded from the thesis, other are required.  LaTeX will
%%  print a warning message if a required entry is not included.

\title{An Adaptive Resolution Criterion for Magnetized Accretion Disk Simulations}

\author{Parth}{Sastry}

\dept{Department of}{Physics}
\program{Physics}
\college{College of}{Engineering}
\conferraldate{January}{2026}
%%
%%  select the degree being earned
%%  \degree{FULL NAME}{ABBREVIATION}
%%
%\degree{Undergraduate}{Undergraduate}
%\degree{Honors}{Honors}
%\degree{Master of Arts}{M.A.}
%\degree{Master of Art Education}{M.A.Ed.}
%\degree{Master of Fine Arts}{M.F.A.}
\degree{Master of Science}{M.S.}

%%  PhD dissertations have extra details such as the program name
%%  and program label to allow for the EAS and BMBMT programs
%%
%\degree{Doctor of  Philosophy}{Ph.D.}
%\phdprogram{PROGRAM NAME}{LABEL}
%%
%%  defined Ph.D. programs -- 2019
%%\phdprogram{Biomedical Engineering and Biotechnology}{BMEBT}
%%\phdprogram{Computational Science}{EAS}
%%\phdprogram{Computer Engineering}{ECE}


%%  advisors and readers have a professorial title, name,
%%  department and affiliation
%%  but NOT degrees earned (ie. do not write Dr. Smith )
%%
%%  eg. \advisor{PROFESSORIAL TITLE}{NAME}{DEPARTMENT}{AFFILIATION}
%%
\advisor{Professor}{Robert Fisher}%
{Physics}{University of Massachusetts Dartmouth}
%%
%%  co-advisors are allowed ....
%%  sometime the co-advisor "replaces" a reader but othertimes there are 
%%  also the full complement of readers. Undergraduate and Masters thesis
%%  have two readers while in the case of Ph.D. dissertations, there are
%%  three readers.  The classfile can cope with all this ambiguity when 
%%  generating the signature page.
%% 
%\coadvisor{Associate Professor}{Fred A. Dagg}%
%{Physics}{University of Massachusetts Dartmouth}

\readerone{Associate Professor}{Sarah Caudill}%
{Physics}{University of Massachusetts Dartmouth}

\readertwo{Assistant Professor}{Vadapalli Vijay Subrahmanya Varma}%
{Mathematics}{University of Massachusetts Dartmouth}

%%  for readers with no academic affiliation
%%  the title, department entres are left empty and the affiliation entry
%%  includes the full address with LaTeX formatting mark-up.

% \readerthree{}{Biggs Darklighter}%
% {}{Imperial Academy\\Tatooine }

%%  the rest of the entries on the signature page have a
%%  professorial title and name (the affiliation will be UMASS DARTMOUTH)

\graddirector{Robert Fisher}
\deptchair{Jianyi Jay Wang}
\collegedean{Robert Griffin}
\gradstudies{Tesfay Meressi}
%%  this is for Honors (undergraduate) theses
%\honorsdirector{Catherine Gardner}

%%
%%  THE ABSTRACT
%%  ============
%%
%%  From the UMass Dartmouth Thesis Guide
%%  "Requirements for Theses and Dissertations" (Spring 2015) 
%%  
%%  5.1.4 Abstract
%%  --------------
%%  The thesis or dissertation must contain an abstract -- a concise summary of
%%  the thesis or dissertation intended to inform a prospective reader about
%%  its content. It usually includes a brief description of the problem
%%  investigated, the procedures or methods used, the findings, and the
%%  conclusions. It may use one or a few paragraphs; however, it is very rare
%%  that an abstract should use more than two pages, and many use just one page. 
%%
\abstract[short]{%
  Accretion is a fundamental source of energy in astrophysics, powering disks in systems ranging from moons to low-mass stellar binaries to disks around supermassive black holes at the centers of galaxies. Accretion is challenging to model accurately from first principles, making numerical simulations a key tool in investigating turbulent magnetohydrodynamic (MHD) fluctuations driven by the magneto-rotational instability (MRI). Resolving the critical length scale associated with this instability is essential to capturing its growth and dynamics. In this work, we utilize the multi physics code FLASH to model accretion disks in idealized two-dimensional simulations and adaptively refine our simulation domain using adaptive mesh refinement (AMR), based upon resolution of the MRI. We then compare the AMR simulations to simulations which uniformly resolve the computational domain everywhere, and thereby showcase the possible computational savings of utilizing AMR in future MRI simulations.
%%  the emply line before the closing brace is REQUIRED to ensure that 
%%  the formating of the abstract page is done correcty
%%  !!DO NOT REMOVE THIS LINE!!

}%
%%  done the abstract !!

%%
%%  THE ACKNOWLEDGEMENTS
%%  ====================
%%
%%  From the UMass Dartmouth Thesis Guide
%%  "Requirements for Theses and Dissertations" (Spring 2015) 
%%  
%%  5.1.5 Acknowledgments
%%  ---------------------
%%  Short statements of acknowledgment of indebtedness (eg. thanks to one's
%%  thesis or dissertation advisor, to other professors, to people who have
%%  given support) may appear on a separate page right after the abstract. An
%%  acknowledgments section is required if the author has received permission to
%%  use previously copyrighted material or is obliged to acknowledge grant
%%  sources. This section is present in most theses or dissertations and is used
%%  to express a very specific professional or personal indebtedness. For
%%  example, significant instances of collaboration with one or more others in
%%  one's thesis or dissertation work would probably need acknowledgment in a
%%  Preface (see 5.1.8) or in this Acknowledgments section for example, research
%%  undertaken together with another student or use of much material from some
%%  other investigator.
%%
%%  The acknowledgements should be written in a professional manner. When
%%  writing the acknowledgments, be sure that your use of "person"� is
%%  consistent.  If you begin with references to yourself as "the author",
%%  continue to use third person throughout. If you begin with first person
%%  ("I", "me", "my") use first person consistently.  There are two accepted
%%  spellings of the word "acknowledgments"� (the other is "acknowledgements");
%%  be sure to spell this word consistently.
%%
\acknowledgements{%
I would like to profusely thank Professor Robert Fisher for guiding me through this work and for being as patient with me as an advisor could be. His encouragement in the face of personal hardship and his guidance as I switched fields from Physics to Oceanography has been invaluable.

I am grateful also to my current advisor - Professor Amit Tandon for extending the opportunity to me to pursue my current Doctor of Philosophy project and for allowing me to pursue this project to its completion, in fact encouraging me to do so. I am supremely thankful for both these amazing mentors I have met on my graduate school journey.

I would also like to thank Professor Sean Couch at Michigan State University who extended to us the opportunity to test out their group's version of the FLASH software in our initial exploration of what simulations were possible for this project. I would also extend my gratitude to the organizers of the NSF/APS-DPP GPAP Summer School in 2023, which was invaluable in my theoretical understanding of the field of plasma physics and magnetohydrodynamics in particular.

I would like to thank Mark Ugalino and Vrutant Mehta for their guidance as I initially got acclimatized to the lab and the software I used, and for being excellent seniors in general. I also extend my sincere and frankly, insufficient, thanks to all the friends I have made in graduate school. I shall name Siddhant Kerhalkar, Debarshi Sarkar, Drake Ssempijja and Krut Patel in particular for their friendship through the journey of writing this thesis.

I am most thankful, however, to my parents, without whose support and love none of this would be possible.

This work utilized resources from Unity, a collaborative, multi-institutional high-performance computing cluster managed by UMass Amherst Research Computing and Data. Part of this work was conducted while I was supported by the Office of Naval Research (ONR) N00014-22-1-2012 grant to UMass Dartmouth MUST (3) program. 
%%  the emply line before the closing brace is REQUIRED to ensure that 
%%  the formating of the acknowledgments page is done correcty
%%  !!DO NOT REMOVE THIS LINE!!

}%
%%  done the acknowledgements !!


%%
%%  THE PREFACE (optional)
%%  ======================
%%
%%  From the UMass Dartmouth Thesis Guide
%%  "Requirements for Theses and Dissertations" (Spring 2015) 
%%  
%%  5.1.8 Preface (optional)
%%  ------------------------
%%  Most theses or dissertations will not have a Preface, which is called for
%%  only for unusual reasons, eg., when the genesis of the work needs to be
%%  explained or when the author's contribution to a multiple-authored work must
%%  be noted. If there is a preface, however, it would incorporate any
%%  acknowledgments instead of those appearing as a separate section.
%%  
%%  Preface sections are rarely used.  The first chapter (sometimes called
%%  "Introduction") in the text section is the appropriate place for
%%  explanations of the context or the motivations that underlie the research,
%%  the research problem, the background of previous scholarship, notable
%%  contributions by other scholars, and so forth. Use a "Preface" section only
%%  for special purposes beyond such purposes as these; examples of such a
%%  special purpose are covered in sections 7.4 and 8.5.
%%  
%%
%%  7.4 Translations by the Author of Material Used
%%  ----------------------------------------------- 
%%  Material that you translate is still the intellectual property of the
%%  author. It must be documented fully (its original-language source cited
%%  properly and included in the bibliography). An appropriate note indicates by
%%  whom it has been translated, by you or someone else. If a thesis or
%%  dissertation will have extensive use of such material, this might be an
%%  occasion for an explanation in a Preface.  Usually, translators of published
%%  works will be indicated in the standard documentation of your notes and/or
%%  bibliography.
%% 
%%  8.5 Collaborative Work That Will Appear in a Thesis or Dissertation
%%  -------------------------------------------------------------------
%%  A thesis or dissertation must represent work done principally if not
%%  entirely by the author. When there are minor instances of research
%%  collaboration, an appropriate citation may be used. If extensive, however,
%%  the committee must approve it in detail and an explanation in a Preface
%%  section is called for (see sections 5.1.8).
%% 
%%  
%%  The preface should be written in a professional manner. When writing the
%%  acknowledgments, be sure that your use of "person" is consistent.  If you
%%  begin with references to yourself as "the author"� continue to use third
%%  person throughout. If you begin with first person ("I", "me", "my"), use
%%  first person consistently.  
%%
%\preface{%
%  The text of the preface goes here.
%%%  the emply line before the closing brace is REQUIRED to ensure that 
%%%  the formating of the preface page is done correcty
%%%  !!DO NOT REMOVE THIS LINE!!
%
%}%
%%%  done the preface !!


%%
%%  THE EPIGRAPH (optional)
%%  =======================
%%
%%  Some authors include a quotation (epigraph) or illustration (frontispiece)
%%  as the last of their preliminary pages. Neither should be listed in the
%%  table of contents, although a frontispiece may be included in the list of
%%  illustrations. The source of an epigraph is indicated below the quotation
%%  but is not listed in the bibliography or references unless it is also
%%  cited in the text. A page number need not be shown, but the page is
%%  counted in the sequential page numbering.

%%  The default option is a epigraph (text).
%%  The two LaTeX commands below will produce the same output.
%%
%\epigraph{%
%    \centering{To all of the fluffy kitties \ldots} 
%%%  the emply line before the closing brace is REQUIRED to ensure that 
%%%  the formating of the preface page is done correcty
%%%  !!DO NOT REMOVE THIS LINE!!
%
%}%
%\epigraph[epigraph]{%
%    \centering{To all of the fluffy kitties \ldots}
%%%  the emply line before the closing brace is REQUIRED to ensure that 
%%%  the formating of the preface page is done correcty
%%%  !!DO NOT REMOVE THIS LINE!!
%
%}%


%%  To include an illustration, use the optional argument of 
%%  frontispiece and the image file name as the second argument
%%
%%\epigraph[frontispiece]{einstein_bike}

%%%  done the epigraph !!


%%  need these if there are no Figure or Tables
%%  otherwise evil things happen to the Table of Contents
%\emptyLoF
%\emptyLoT

%%  uncomment to aviod generating prologue pages
%\SuspendPrologue

%%  All the prologue pages are done.  The thesis proper begins after here.

%%
%%  End of LaTeX preamble
%%  =========================================================================

\begin{document}%

%%  set the format required for the citations/references
%%  \bibliographystyle{unsrt} is preferred for UMassD theses & dissertations.
\bibliographystyle{aasjournal}

%%  The document text can be typed directly into this file or make use of
%%  the LaTeX \input{filename} command to read the contents of the file
%%  filename.tex.
%%  The later method is a good way to logically organize the material.

%%  INSERT THESIS BODY HERE

%%  Using the LaTeX \input command 
\chapter{Introduction}
\label{sec:intro}

\pagenumbering{arabic} 

Astrophysical systems are inherently multi-scale, with physically relevant processes often occurring at length scales that span many orders of magnitude. This is readily apparent in astrophysical plasmas in a variety of contexts - the interstellar medium (ISM), protoplanetary and accretion disks, the solar corona, and molecular clouds. With the rise of computing capabilities, numerical simulations have become an important tool in understanding the evolution of these systems. Capturing the essential dynamics of the system necessarily entails that simulations incorporate physical processes either by fully resolving the requisite length scales in a direct numerical simulation approach, or otherwise capturing them via subgrid models. In this thesis, we will focus specifically upon direct numerical simulations. 

%Due to the inherently multi-scale nature of such systems, 
Dynamically relevant length scales will generally exhibit spatial variation over the simulation domain as well as temporal variation as the simulation evolves. A conservative approach to capturing these relevant length scales involves uniformly resolving the smallest length scale across the entire simulation. However, this approach is generally computationally expensive, with some regions of the simulation domain often over-resolved at some time steps. For example, in the context of self-gravitational hydrodynamics, it was found that numerical instabilities can be avoided by ensuring that the Jeans length is always resolved \citep{1997ApJ...489L.179T}. %number $J \equiv \Delta x/\lambda_J \leq 0.25$, called the Jeans condition. Here, $\lambda_J = ( \pi c_s^2 / G \rho )^{1/2}$ is the Jeans length (\cite{1902RSPTA.199....1J}, \cite{Jeans1928-JEAAAC}).
Because the Jeans length is not a constant across space and time, simulations that locally resolve the Jeans criterion using adaptive mesh refinement, smoothed particle hydrodynamics, or a moving mesh are vastly less computationally expensive than maximally refining the Jeans length everywhere and all times, while also ensuring numerical stability and convergence \citep{hummels_adaptive_2012}.
% This text is redundant
%Adaptive Mesh Refinement (AMR) simulations offer an alternative to this by adaptively refining the grid based upon a relevant criterion.

This paradigm of tracking characteristic length scales is applicable to a wide range of physical processes, and has been applied for example to thermal conduction instabilities \citep{koyamainutsuka04, gressel09}, magnetized self-gravitating turbulent dynamos \citep{federrathetal11}, and gaseous cooling \citep{reyetal24}.
%It was shown in \cite{1965ApJ...142..531F} that there exists a stability criterion for a dilute gas in thermal and mechanical equilibrium and that a near-isentropic sound wave can grow in amplitude if an instability condition is satisfied, causing condensations of density. If this is the relevant dynamical phenomenon in the system, then tracking the fastest growing unstable mode of this instability via AMR simulations should suffice to capture essential dynamics of the system.
In this thesis, we explore length scale tracking in the context of the magnetorotational instability (MRI, \cite{1991ApJ...376..214B}, \cite{RevModPhys.70.1}) in accretion disks, where the MRI is generally understood to be a primary driver of disk turbulence and disk evolution, causing radially outward angular momentum transfer and inward mass accretion through the action of the combined Reynolds and Maxwell turbulent stresses.

Because of its key role in driving angular momentum transport, it has long been understood that resolving the fastest-growing MRI mode $\lambda_{\text{MRI}}$ is important to accurately capture accretion disk dynamics in MHD simulations. For example, through shearing box simulations, \citet{2004ApJ...605..321S} found that to resolve MHD turbulence driven by the MRI, one needs to have $\langle \langle \lambda_{\text{MRI}}^2 \rangle \rangle^{1/2} \gtrsim 6 \Delta$, where $\Delta$ is the grid length, and the double brackets indicate a volume and time average. Simulations that meet this criteria accurately capture the linear growth of the fastest-growing mode as well as the saturation limits of the magnetic field.

In an axisymmetric geometry, the MRI wavelength $\lambda_{\text{MRI}}$ is determined from the vertical magnetic field component $B_z$ \citep{1991ApJ...376..214B}, i.e.,

\begin{equation} \label{eqn:lambdamri}
     \langle \lambda_{\text{MRI}}^2 \rangle^{1/2} = 2\pi \frac{\langle v_{\text{A}z}^2 \rangle^{1/2}}{\Omega} = \frac{2\pi}{\Omega} \left( \frac{\langle B_{z}^2 \rangle }{4 \pi \rho} \right)^{1/2}
\end{equation}
%
Here $v_{\text{A}z}$ is the Alfv\'en velocity derived from the vertical magnetic field, $\Omega$ is the local disk angular velocity, and $\rho$ is the mass density.


An initial seed magnetic field undergoes exponential growth via the MRI, eventually saturating when the fastest growing mode becomes of order the disk scale height, $\lambda_{\text{MRI}} \sim h \sim \Omega / c_s$. Frequently, the seed magnetic field for the growth of the MRI is orders of magnitude weaker than its final saturation strength, implying that one needs to capture a dynamic range $h / \lambda_{\text{MRI}, 0} \sim 2 \pi c_s / v_{A,0}$, where $\lambda_{\text{MRI}, 0}$ and $v_{A,0}$ are the fastest growing mode length and Alfv\'en speed based upon the initial seed magnetic field. For example, \citet{kiuchi_large-scale_2024} found that in binary neutron star mergers, an initial seed field of $\sim 10^{44}$ erg at time of merger increases by roughly 4 orders of magnitude to $\sim 10^{48}$ erg within roughly 2 ms after merger.

Consequently, a key challenge to numerical simulations is to be able to capture both the growth of the magnetic field and the development of disk turbulence under the action of the MRI, while also simultaneously resolving the global disk structure. One must either use extremely high resolution to capture the fastest growing MRI mode of the weak seed magnetic field, or otherwise utilize an artificially strong magnetic field.  

In Section \ref{sec:simulation}, we present our  proposed AMR criterion for MRI disks, as well as the numerical simulations for verification. We also discuss the effect of simulation dimensionality on the proposed criterion and its extension to more general $3$D simulations. In Section \ref{sec:results}, we compare simulations utilizing the proposed AMR criterion and reference uniform grid simulations via a variety of diagnostics to qualitatively and quantitatively assess closeness of simulations. In Section \ref{sec:discuss}, we discuss some of the implications of these results, follow-up on our note on simulation dimensionality with some discussion of the extension of the AMR criterion to $3$D and conclude our work. 
\chapter{Simulation Setup}
\label{sec:simulation}

\pagenumbering{arabic} 

Our simulation setup is similar to that of the GT2 global torus setup used by \citet{2000ApJ...528..462H}. We solve the equations of non-relativistic ideal MHD in axisymmetric $2.5$D cylindrical coordinates $(R, \phi, z)$ (see \ref{subsec:dimension} for notes on simulation dimensionality). Written in non-dimensional form (without $4\pi$ and $\mu_0$ coefficients), these are

\begin{equation}\label{eqn:mass}
    \frac{\partial \rho}{\partial t} + \nabla \cdot (\rho \vec{v}) = 0
\end{equation}
\begin{equation}\label{eqn:momentum}
    \frac{\partial \rho\vec{v}}{\partial t} + \nabla \cdot \left( \rho \vec{v}\vec{v} - \vec{B}\vec{B}\right) + \nabla p_{*} = -\rho \nabla \Phi
\end{equation}
\begin{equation}\label{eqn:energy}
    \frac{\partial \rho E}{\partial t} + \nabla \cdot \left(\vec{v} \left( \rho E + p_{*} \right) - \vec{B}\left(\vec{v} \cdot \vec{B} \right) \right) = -\rho \vec{v} \cdot \nabla\Phi
\end{equation}
\begin{equation}\label{eqn:mag}
    \frac{\partial \vec{B}}{\partial t} = \nabla \cdot ( \vec{B}\vec{v} - \vec{v}\vec{B} )
\end{equation}

where

\begin{equation}\label{eqn:defp*}
    p_* = p + \frac{B^2}{2}
\end{equation}
\begin{equation}\label{eqn:defE}
    E = \frac{1}{2}v^2 + \epsilon + \frac{1}{2}\frac{B^2}{\rho}
\end{equation}
%
%and the symbols have their usual meanings.

Here $\rho$ is the mass density, $\vec{v}$ is the fluid velocity, $p$ is the fluid pressure, $\vec{B}$ is the nondimensional magnetic field vector, $\epsilon$ is the specific internal energy, and $\Phi$ is the gravitational potential.

The equations used above utilize Heaviside-Lorentz normalization. The magnetic field vector as mentioned in the system of equations above is related to the magnetic field vector in the cgs system of units as -

\begin{equation}
    \vec{B} = \frac{\vec{B}_{\rm cgs}}{\sqrt{4\pi}}
\end{equation}

We utilize an adiabatic equation of state $p = \rho \epsilon( \Gamma - 1)$ with $\Gamma = 5/3$, and as in \citet{2000ApJ...528..462H}, ignore radiation transport and losses. The Reynolds Number $\mbox{\textit{Re}} = uL/\nu$ in astronomical accretion disks is massive (of the order of $\sim 10^{10}$) and thus, we do not include a molecular viscosity term in equations \ref{eqn:momentum} and \ref{eqn:energy}. This is backed by analysis in \citet{1974MNRAS.168..603L} and \citet{1981ARA&A..19..137P}, which concludes that molecular viscosity plays a negligible role in angular momentum transport outwards and thus, the dynamical evolution of the accretion disk.

To approximate the gravitational potential of a compact, non-rotating object, we use a pseudo-Newtonian potential \citep{1980A&A....88...23P}.

\begin{equation}
    \Phi = - \frac{GM}{r - r_g}
\end{equation}
%
where $r$ is the spherical radius and $r_g$ is the "gravitational radius". The Keplerian specific angular momentum for such a potential is given by

\begin{equation}
    l = (GMr)^{1/2} \frac{r}{r - r_g}
\end{equation}
%
with $\Omega = l/R^2$, $R$ being the cylindrical radius. As in \citet{2000ApJ...528..462H}, we use simulation units where $r_g = 1$ and $GM = 1$, and excise the center of the cylindrical coordinate system by setting an outflow boundary condition at $R = 1.5$, which avoids singularities associated with $\Phi$ at $r \rightarrow 1$. A snapshot of the computational domain and the initial conditions used for the torus is shown in Figure \ref{fig:init-density}.

Equations \ref{eqn:mass}, \ref{eqn:momentum}, \ref{eqn:energy} and \ref{eqn:mag} are solved using a directionally unsplit, staggered mesh solver \citep{2009JCoPh.228..952L, 2013JCoPh.243..269L}, implemented in the multiphysics code FLASH 4.2 \citep{2000ApJS..131..273F, 2008PhST..132a4046D}. We use second order data reconstruction to compute the cell-interface values in the half time step and use the HLLC (Harten-Lax-van Leer-Contact) Riemann solver \citep{1994ShWav...4...25T} as implemented in FLASH.

\citet{2013ApJ...772..102H} explored convergence of global disk simulations by looking at systematic quantities of interest like the quality factors $\langle Q_z \rangle$, $\langle Q_\phi \rangle$ and quantities that parametrize the accretion rate and the stress development due to turbulence, $\langle \alpha_\text{SS} \rangle = \tau_{r\phi}/P$ (the Shakura-Sunyaev alpha \citep{1973A&A....24..337S}) and $\alpha_\text{mag} = \tau_{r\phi}/P_\text{mag}$. While our $2$D axisymmetric simulations do not have sustained MHD turbulence and do not capture toroidal field amplification due to the MRI (see \ref{subsec:dimension} for more on dimensionality), these quantities, along with global quantities like the poloidal magnetic energy, total disk mass, accretion rate, etc. offer a quantitative way to assess the closeness of the AMR simulations with the maximally refined uniform grid simulation, which is taken to be the fiducial run for our simulations.

\section{Initial Conditions} \label{subsec:init-cond}

We are concerned with the dynamical evolution of accretion tori subject to the magnetorotational instability (MRI). As such, the initial torus has to be hydrostatically stable, with internal pressure gradients balancing out the gravitational and centrifugal acceleration terms. The initial torus is assumed to be an axisymmetric, constant specific angular momentum torus as described in \citet{1984MNRAS.208..721P}. This describes the density of the torus as a function of position

\begin{equation}\label{eqn:pptorus}
    \frac{\Gamma K p}{(\Gamma - 1)\rho} = C - \Phi - \frac{1}{(2q - 2)}\frac{l^2}{R^{2q-2}}
\end{equation}

where, for the case of constant specific angular momentum, $q = 2$. $C$ is a constant of integration defining the zero pressure boundary of the torus, which can be set via the inner boundary of the torus, $R_{\text{in}}$. The only other parameter we need to fully describe the torus is the location of the pressure maximum, which \citet{2000ApJ...528..462H} defines as $R_{\text{Kep}}$, which is the distance at which an isolated particle with the given constant specific angular momentum would rotate in a circular orbit.

Our simulations are conducted in $2.5$-D axisymmetry. The computational domain extends from $(1.5, 15.5)$ in $R$ and $(-7, 7)$ in $z$. We set the inner boundary of the torus $R_{\text{in}} = 3$ and $R_{\text{Kep}} = 4.7$ in code units. The value of the constant $K$ is set via defining the value of the density at the density maximum $\rho_{\text{max}} = 10$.

Our boundary conditions along the $R$ and $z$ boundaries are outflow and periodic in $\phi$. With $2.5$-D geometry, we utilize a single grid cell to cover the entire $(0, 2\pi)$ domain in $\phi$, which basically imposes axisymmetry at all times. This makes sure that we can look at the evolution of quantities in the $\phi$-direction, such as the toroidal magnetic field component $B_\phi$, but these quantities remain constant across $\phi$.

The hydrodynamics solver in FLASH requires some floor value of density on the entire grid. Thus, the region outside the torus is composed of a hot, low density "fluff" material. This fluff has to have low density so as to allow almost free expansion and outflow of the torus material through the computational domain.. However the density cannot be too low, otherwise we will get timestep-limited by the fluff, owing to the low sound-crossing times in the fluff. The density profile of the fluff is set via assuming a density contrast of $10^{-4}$ between the torus maximum density and the fluff density, leading to a fluff density of the order of $10^{-3}$.

The ambient pressure of the fluff is in pressure balance with the torus at the outer edge. The density transition between the torus and the fluff is smooth and the fluff is assumed to match the composition of the torus, to avoid composition gradients at the boundaries. A snapshot of the initial density profile across the computational domain is given in Figure \ref{fig:init-density}

For the magnetic field, we utilize two initial field configurations with two different average field strengths for our simulations. The first is a standard dipole loop, hereafter called the "single loop" configuration, where the vector potential is of the form

\begin{equation}\label{eqn:dipolefield}
    A_{\phi} = A_0(\rho - \rho_{\text{cut}})
\end{equation}

The second field configuration is the "double loop" configuration as described by \citet{2008ApJ...687L..25S}. The vector potential takes the form

\begin{equation}\label{eqn:doubleloop}
    A_{\phi} = A_0 \left[ \left( \rho - \rho_{\text{cut}} \right) r^{0.75} \right]^2 \sin{\left[ \ln{(r/S)}/T \right]}
\end{equation}

$\rho_{\text{cut}}$ defines the density cut-off beyond which the magnetic field is identically zero, i.e., $A_0 = 0$, if $\rho < \rho_{\text{cut}}$. $r$ is the spherical radius, $S = 1.1 R_\text{in} = 3.3$ and $T = 0.16$. For both the configurations, our simulations utilize a $\rho_\text{cut} = 0.2\rho_\text{max}$, thereby constraining the poloidal field loops well within the torus boundary.  The simulations are labelled singleloop and doubleloop respectively.

The constant $A_0$ is set via an initial value of the plasma-$\beta$ in the torus, which is the ratio of the volume-integrated gas pressure to the volume-integrated magnetic pressure within the torus. We use $\beta = 100, 1000$ for the singleloop simulations and $\beta = 1000$ for the doubleloop simulations.  These field configurations have an impact on resolution of the MRI and the particular AMR criterion used for refinement, which we will discuss in \ref{subsec:grid}

\section{Grid Structure} \label{subsec:grid}
For each of the field configurations and $\beta$ values mentioned above, we perform five sets of simulations which vary in the structure of the computational grid. There are three uniform grid simulations, with grid resolutions of $N_R \times N_z = 512 \times 512$ ($\Delta R = \Delta z = 0.027344$), $N_R \times N_z = 1024 \times 1024$ ($\Delta R = \Delta z = 0.013672$) and $N_R \times N_z = 2048 \times 2048$ ($\Delta R = \Delta z = 0.006836$) hereafter referred to as UG-level7, UG-level8 and UG-level9 simulations respectively. Here, $N_R$ and $N_z$ refer to the number of grid points in the $R$ and $z$ directions over the entire domain.

The final two simulations are performed using an adaptively refined grid, which have resolutions varying from $N_R \times N_z = 64 \times 64$ ($\Delta R = \Delta z = 0.21875$) at the lowest level (referred to as level 4) to $N_R \times N_z = 1024 \times 1024$ ($\Delta R = \Delta z = 0.013672$) for the AMR-lmax8 simulation and $N_R \times N_z = 2048 \times 2048$ ($\Delta R = \Delta z = 0.006836$) for the AMR-lmax9 simulation. These simulations should be compared against the uniform grid simulations corresponding to their maximal resolution. See Table \ref{tab:sims} for a summary of all simulations.

The AMR package used in FLASH is PARAMESH \citep{2000CoPhC.126..330M}, which uses a block-structured AMR scheme (see also \citet{1984JCoPh..53..484B}, \citet{1989JCoPh..82...64B} and \citet{1993JCoPh.104...56D}). All cells within a block exist at the same refinement level, and adjacent blocks may differ only by a single refinement level. The refinement level of a block is contingent upon a defined criterion. For our purpose of looking at the dynamical evolution of an accretion torus, we chose to refine by looking at the quality factor $Q_z$ within the block.

%Moved to intro
%Because the magnetorotational instability (MRI) is the driver of angular momentum transport, and thus what drives the dynamical evolution of the torus, it has long been understood that resolving the fastest-growing MRI mode is important for simulations to accurately capture disk dynamics. Through shearing box simulations, \cite{2004ApJ...605..321S} found that to resolve MHD turbulence driven by the MRI, one needs to have $\langle \langle \lambda_{\text{MRI}}^2 \rangle \rangle^{1/2} \gtrsim 6 \Delta$, where $\Delta$ is the grid size, and the double brackets indicate a volume and time average. Simulations that meet this criteria accurately capture the linear growth of the fastest-growing mode as well as the saturation limits of the magnetic field.

%The MRI wavelength $\lambda_{\text{MRI}}$ is calculated from the vertical component, because the fastest growing axisymmetric mode is characterized by the vertical magnetic field (\cite{1991ApJ...376..214B}), i.e.,

%\begin{equation} \label{eqn:lambdamri}
%    \langle \lambda_{\text{MRI}}^2 \rangle = 2\pi \frac{\langle v_{\text{A}z}^2 \rangle}{\Omega} = \frac{2\pi}{\Omega} \left( \frac{\langle B_{z}^2 \rangle }{4 \pi \rho} \right)^{1/2}
%\end{equation}

%This is the same criteria we apply to the Adaptive Mesh Refinement routine in our simulations. 

Our initial field configurations are those consisting of poloidal field loops, which means that there are magnetized regions in the torus where we cannot resolve $\lambda_\text{MRI}$ (regions where $B_z \rightarrow 0$). In the alternate scenario of a uniform vertical magnetic field $B_z$ threading the torus, it is possible to resolve $\lambda_\text{MRI}$ throughout the torus at the initial time step, but as do other modern simulations of accretion tori \citep{2024ApJ...973..103C, 2019MNRAS.482.3373F}, we initialize the magnetic field with these poloidal field loops, representing a more physical scenario.

We initially describe a mass scalar $\chi$, which is a field variable advected with density, and set this field to a value of $1$ inside the torus and $0$ outside, in the fluff. We create this separation by setting the boundary between the torus and the fluff at a threshold density of $10^{-4}$ times $\rho_\text{max}$ (see Figure \ref{fig:init-density}). We then advect this scalar field according to

\begin{equation}\label{eqn:masscalar}
    \frac{\partial \rho \chi}{\partial t} + \nabla \cdot \left( \rho \chi \vec{v} \right) = 0
\end{equation}

\begin{figure}[ht!]
    \centering
    \includegraphics[width=0.8\textwidth]{figures/density.pdf}
    \caption{Initial density profile of the torus, with a red contour line demarcating the boundary between the disk ($\chi = 1$) and the fluff ($\chi = 0$)}
    \label{fig:init-density}
\end{figure}

At later times, the value of the mass scalar $\chi$ represents how much of a grid cell's contents are composed of material from the initial torus. Since we are mainly concerned with the disk dynamics, we set a threshold value $\chi_{\text{min}} = 0.1$. If a block doesn't contain any cells with $\chi > \chi_{\text{min}}$, that block will not be refined. For blocks containing cells with $\chi > \chi_{\text{min}}$, we look at the quality factor $Q_z$ of those cells

\begin{equation}\label{eqn:qz}
    Q_z = \frac{\lambda_{\text{MRI}}}{\Delta z} = 2\pi \frac{\vert v_{\text{A}z} \vert}{\Omega\Delta z}
\end{equation}

If there exists a grid cell within the block with $Q_z < 6$, we increase the refinement level of the block, doubling the resolution of the block (and increasing the number of cells within the block by a factor of $2^D$ where $D$ is the simulation dimensionality. For our simulations $D = 2$).

A slice plot highlighting how our MRI-AMR criterion refines the grid at the initial time step is shown in Figure \ref{fig:grid} for the doubleloop-beta1000 simulation. This is plotted for the AMR-lmax9 simulation. Because the field is constrained within a field loop initially, we have to maximally refine regions of the torus with $\rho < \rho_\text{cut}$ (outside the field loop, but within the torus boundary), where $B_z$ and thus, $\lambda_\text{MRI}$ is 0.

\begin{figure}
    \centering
    \begin{subfigure}[t]{0.8\textwidth}
        \centering
        \captionsetup{width=\textwidth}
        \includegraphics[width=0.9\textwidth]{figures/AMRlmax9_grid.pdf}
        \caption{\footnotesize Block structure as defined by our resolution criteria at the initial time step of the doubleloop-beta1000 simulation, for a maximum resolution level of 9, overlaid on the poloidal magnetic field magnitude. Note that each block is an $8 \times 8$ grid of cells.}
        \label{subfig:doubleloop-beta1000-fullGrid-AMRlmax9}
    \end{subfigure}%

    \begin{subfigure}[t]{0.49\textwidth}
        \centering
        \captionsetup{width=0.9\textwidth}
        \includegraphics[width=0.9\textwidth]{figures/level9_grids.pdf}
        \caption{\footnotesize Grid structure at the edge of the torus in the UG-level9 simulation. Note that the individual unit in this figure is a single cell.}
        \label{subfig:UG-level9-torusedge}
    \end{subfigure}
    \begin{subfigure}[t]{0.49\textwidth}
        \centering
        \captionsetup{width=\textwidth}
        \includegraphics[width=0.9\textwidth]{figures/AMRlmax9_grids.pdf}
        \caption{\footnotesize Grid structure at the edge of the torus in the AMR-lmax9 simulation, highlighting lower refinement outside the torus.}
        \label{subfig:AMR-lmax9-torusedge}
    \end{subfigure}
    \vspace{-1em}
    \caption{Grid structure}
    \vspace{-2.5em}
    \begin{flushright}
    \footnotesize
        (cont. on next page)
    \end{flushright}
\end{figure}

\begin{figure}[ht!]\ContinuedFloat
    \centering
    \begin{subfigure}[t]{0.5\textwidth}
        \centering
        \captionsetup{width=1.5\textwidth}
        \includegraphics[width=\textwidth]{figures/AMRlmax9_zoomedgrid.pdf}
        \caption{\footnotesize Grid structure at center of the torus in the AMR simulation, highlighting lower refinement in regions of the torus where the magnetic field is largely vertical and the quality factor is high.}
        \label{subfig:AMR-lmax9-toruscenter}
    \end{subfigure}
    \caption{(cont.)}
    \label{fig:grid}
\end{figure}

\begin{table}[hbt!]
\centering
\caption{Summary of simulations performed}
\label{tab:sims}
\normalsize
\begin{tabular}{c|c|c|c|c}\toprule
       Simulation & Field Type & 
       $\langle \beta \rangle_\text{initial}$ &
       Resolution Class &
       $N_R = N_z$ \\\midrule
        \multirow{5}{*}{\texttt{doubleloop-beta1000}} & \multirow{5}{*}{Double loop} & \multirow{5}{*}{1000} & \texttt{UG-level7} & 512 \\
        & & & \texttt{UG-level8} & 1024 \\
        & & & \texttt{UG-level9} & 2048 \\
        & & & \texttt{AMR-lmax8} & 64-1024 \\
        & & & \texttt{AMR-lmax9} & 64-2048 \\ 
        \hline
        \multirow{5}{*}{\texttt{singleloop-beta1000}} & \multirow{5}{*}{Dipole one-loop} & \multirow{5}{*}{1000} & \texttt{UG-level7} & 512 \\
        & & & \texttt{UG-level8} & 1024 \\
        & & & \texttt{UG-level9} & 2048 \\
        & & & \texttt{AMR-lmax8} & 64-1024 \\
        & & & \texttt{AMR-lmax9} & 64-2048 \\
        \hline
        \multirow{5}{*}{\texttt{singleloop-beta100}} & \multirow{5}{*}{Dipole one-loop} & \multirow{5}{*}{100} & \texttt{UG-level7} & 512 \\
        & & & \texttt{UG-level8} & 1024 \\
        & & & \texttt{UG-level9} & 2048 \\
        & & & \texttt{AMR-lmax8} & 64-1024 \\
        & & & \texttt{AMR-lmax9} & 64-2048 \\\bottomrule
\end{tabular}
\end{table}

\section{Simulation Dimensionality} \label{subsec:dimension}

Our simulations are conducted in "$2.5$D" axisymmetric cylindrical geometry, which means that there are the usual three nonzero field components (in $R$, $\phi$ and $z$ directions) for the vectors in equations \ref{eqn:mass}, \ref{eqn:momentum}, \ref{eqn:energy} and \ref{eqn:mag} ($\vec{v}$ and $\vec{B}$), but the azimuthal derivative $\partial/\partial\phi$ of any field quantity is zero.

While not affording the same level of accuracy as three-dimensional simulations, the axisymmetric approximation tracks the three-dimensional simulations quite well, especially in the initial stages of torus evolution, which are dominated by axisymmetric dynamics like toroidal field generation due to shear and linear growth of the poloidal MRI (\cite{2000ApJ...528..462H}).

A limitation of the axisymmetric approximation is that the initial poloidal field cannot be maintained over long periods due to Cowling's antidynamo theorem \citep{1933MNRAS..94...39C}. After the MRI in our simulations is saturated, the poloidal field begins to die out, and the magnetic energies decline.

One approach to capturing accurate long-term accretion disk evolution even in axisymmetry in an attempt to sidestep Cowling involves incorporating a subgrid mean field dynamo calibrated to capture 3D dynamics \citep{sadowskietal15}. However, accurately calibrating the mean field dynamo subgrid model is challenging, and is the subject of ongoing work \citep{PhysRevD.111.023040}. 

Because three-dimensional simulations couple the poloidal and toroidal fields, along with growing the non-axisymmetric modes of the MRI, there ceases to exist a single length scale of interest that we can track and the AMR criterion will have to change according to what convergence studies deem appropriate levels of resolution. 

Comments on the resolution and the AMR criteria required to capture nonlinear effects in MHD turbulence and growth of the non-axisymmetric MRI modes in $3$D MHD are out of the scope of this thesis. We solely seek to explore whether scaling the resolution of $2$D axisymmetric simulations via looking at $Q_z$ captures disk dynamics at reasonable levels of accuracy compared to the corresponding uniform grid simulations. However, based on the convergence of the AMR simulations to the uniform grid simulations in this thesis, it is likely that AMR simulations where we scale the refinement level depending on resolutions that convergence studies deem necessary would reasonably track the highest resolution uniform grid simulations even in three dimensions.
\chapter{Results}
\label{sec:results}

We have five sets of simulations varying in their grid structure for a few initial field configurations and strengths. A summary of these is presented in Table \ref{tab:sims}.

For the figures shown below, the appropriate comparisons to make are among the AMR and the corresponding uniform grid simulations, i.e among the AMR-lmax9 and the UG-level9 simulations since both those simulations have an equivalent maximum resolution ($\Delta R = \Delta z = 0.006836$). Likewise with the AMR-lmax8 and the UG-level8 simulations where the maximum resolution is $\Delta R = \Delta z = 0.013672$. The UG-level7 simulations are given as a lower limit on what underresolved simulations can look like.

We run our simulations from time $t = 0$ to $t = 300$, corresponding to 8 orbits at the pressure maximum. We do not extend the simulations further due to observed instabilities in the highest resolution simulations once we enter the phase of evolution of the disk where we encounter MRI turbulence and the poloidal MRI is decaying.

\cite{2000ApJ...528..462H} found, in their axisymmetric $2$D simulations with a single poloidal loop, that the poloidal MRI sets in around $t = 150$ and a period of turbulence follows, after which the poloidal field begins to die out. Thus, our simulations capture the period of the simulation during which we expect our adaptive resolution criterion to capture disk dynamics accurately - the growth phase of the poloidal MRI, and the shear amplification of the toroidal field.

A slice of the simulation domain comparing density and the magnitude of the poloidal magnetic field at the final time, $t = 300$ in the simulation is shown for the singleloop-beta1000 simulation in Figure \ref{fig:slice}.

While there are differences in evolution among all the different resolution classes - even among those that are meant to be comparable - of note is the similarity in the poloidal magnetic field magnitude and structure within the torus among comparable simulations.

This is also borne out in the evolution of the disk poloidal magnetic energy with time, shown in global quantity plots in subsection \ref{subsec:global} - comparable simulations show the same MRI growth characteristics and the maximum field amplitude and energy.

\begin{figure}[ht!] 
    \centering
    \begin{subfigure}{0.49\textwidth}
        \centering
        \captionsetup{width=0.9\textwidth}
        \includegraphics[width=\linewidth]{figures/singleloop-beta1000_l9_t-0400.pdf}
        \caption{UG-level9.}
        \label{fig:slice1}
    \end{subfigure}
    \begin{subfigure}{0.49\textwidth}
        \centering
        \captionsetup{width=0.9\textwidth}
        \includegraphics[width=\linewidth]{figures/singleloop-beta1000_amrl9_t-0400.pdf}
        \caption{AMR-lmax9. Note the similar magnetic field structure and magnitude.}
        \label{fig:slice2}
    \end{subfigure}

    \begin{subfigure}{0.49\textwidth}
        \centering
        \captionsetup{width=0.9\textwidth}
        \includegraphics[width=\linewidth]{figures/singleloop-beta1000_l8_t-0400.pdf}
        \caption{UG-level8. The lack of MRI growth in initially weakly magnetized sections of the torus is apparent.}
        \label{fig:slice3}
    \end{subfigure}
    \begin{subfigure}{0.49\textwidth}
        \centering
        \captionsetup{width=0.9\textwidth}
        \includegraphics[width=\linewidth]{figures/singleloop-beta1000_amrl8_t-0400.pdf}
        \caption{AMR-lmax8. Similar qualitatively and quantitatively to the UG-level8 simulation.}
        \label{fig:slice4}
    \end{subfigure}

    \begin{subfigure}{0.49\textwidth}
        \centering
        \captionsetup{width=\textwidth}
        \includegraphics[width=\linewidth]{figures/singleloop-beta1000_l7_t-0400.pdf}
        \caption{UG-level7. Note the lack of poloidal field amplification within the body of the torus.}
    \end{subfigure}
    \vspace{-1.5em}
    \caption[Density and poloidal magnetic field magnitude slices at the end of the singleloop-beta1000 simulation.]{Density and poloidal magnetic field magnitude slices at the end of the singleloop-beta1000 simulation. Density is on the left and poloidal magnetic field magnitude is on the right.}
    \label{fig:slice}
\end{figure}

\section{Global Quantities} \label{subsec:global}

A useful quantitative measure of the progress of an accretion disk simulation is the amount of mass left on the grid, quantifying the amount of accretion that has occurred on to the central object (and mass lost due to outflow). This is shown in Figure \ref{fig:doubleloop-beta1000mass} for the doubleloop-beta1000 simulation, Figure \ref{fig:singleloop-beta1000mass} for the singleloop-beta1000 simulation, and Figure \ref{fig:singleloop-beta100mass} for the singleloop-beta100 simulation as a fraction of the initial mass on the grid. 

\begin{figure}
    \centering
    \includegraphics[width=0.5\linewidth]{figures/doubleloop-beta1000_mass.pdf}
    \caption{Total mass on the simulation domain for the doubleloop-beta1000 simulation}
    \label{fig:doubleloop-beta1000mass}
\end{figure}

\begin{figure}
    \centering
    \includegraphics[width=0.5\linewidth]{figures/singleloop-beta1000_mass.pdf}
    \caption{Total mass on the simulation domain for the singleloop-beta1000 simulation}
    \label{fig:singleloop-beta1000mass}
\end{figure}

\begin{figure}
    \centering
    \includegraphics[width=0.5\linewidth]{figures/singleloop-beta100_mass.pdf}
    \caption{Total mass on the simulation domain for the singleloop-beta100 simulation}
    \label{fig:singleloop-beta100mass}
\end{figure}

As expected, graphs for the total mass left on the grid for the comparable AMR and UG simulations show similar characteristics. Very notably, onset of accretion happens around the same time and the steady mass loss rate is similar for comparable simulations. Also as expected, the rate of the aforementioned mass loss is highest for simulations where the MRI is well-resolved and lowest for the UG-level7 simulation, where the MRI is least resolved. It is also apparent in Figure \ref{fig:singleloop-beta100mass} that the lowest resolution simulation saturates the MRI sooner as well, stopping accretion sooner than the other simulations.

Another quantity relevant to the evolution of the MRI and the progress of the simulation is the poloidal magnetic energy, shown in Figures \ref{fig:doubleloop-beta1000poloidalmag}, \ref{fig:singleloop-beta1000poloidalmag} and \ref{fig:singleloop-beta100poloidalmag}. The maximum energy before decay provides an indicator of the saturation field, while the growth profile shows the growth of the poloidal MRI. While not relevant to where this adaptive resolution criterion should be used - during the growth phase of the poloidal MRI - interesting to note is that the decay of the poloidal field also matches well among the AMR and the corresponding uniform grid simulations. Of note is Figure \ref{fig:doubleloop-beta1000poloidalmag}, where the decay of the poloidal field agrees among comparable simulations.

\begin{figure}
    \centering
    \includegraphics[width=0.5\linewidth]{figures/doubleloop-beta1000_poloidal_magnetic_energy.pdf}
    \caption{Disk poloidal magnetic energy for the doubleloop-beta1000 simulation}
    \label{fig:doubleloop-beta1000poloidalmag}
\end{figure}

\begin{figure}
    \centering
    \includegraphics[width=0.5\linewidth]{figures/singleloop-beta1000_poloidal_magnetic_energy.pdf}
    \caption{Disk poloidal magnetic energy for the singleloop-beta1000 simulation}
    \label{fig:singleloop-beta1000poloidalmag}
\end{figure}

\begin{figure}
    \centering
    \includegraphics[width=0.5\linewidth]{figures/singleloop-beta100_poloidal_magnetic_energy.pdf}
    \caption{Disk poloidal magnetic energy for the singleloop-beta100 simulation}
    \label{fig:singleloop-beta100poloidalmag}
\end{figure}

As \cite{2000ApJ...528..462H} observed in their corresponding singleloop-beta100 simulations, so do we note a period of poloidal field amplification due to the MRI from the beginning of the simulation, followed by a period of turbulence and decay of the poloidal field.

Interesting to note is also the effect of initial magnetic field structure on the evolution of the simulation. As \cite{2011ApJ...738...84H} observed in their convergence studies, the two-loop simulations reach a higher poloidal magnetic energy and evolve faster than the corresponding single loop simulation. This is to be expected, initial field geometry - despite the same initial $\beta$ - leads to the one loop configuration having an initially weaker $B_z$, which leads to slower MRI growth and slower disk evolution. Thus, the doubleloop-beta1000 simulation showcases the growth of the MRI, subsequent short-lived MRI turbulence and decay all within the time of the simulation.

We see that the AMR simulations scaling with $\lambda_\text{MRI}$ with a maximum resolution equivalent to the corresponding uniform grid simulations track global quantities like total mass and the poloidal magnetic energy (which is a measure of how the poloidal MRI has developed) about as well as the uniform grid simulations do, and better than the lower resolution uniform grid simulations - which seems to suggest that global disk properties are captured well with an adaptive resolution criterion as described, particularly during the phase of disk evolution that our simulations run over.


\section{Radial Profiles} \label{subsec:radial}

While useful, global disk quantities do not state anything about the spatial variation in relevant quantities. For this, we analyze the accretion stream in each of the simulations and compare the radial profiles of quantities of interest. 

Radial profiles are computed via a $z-$average of the quantity in question within the disk midplane. For instance, if we were to look at any quantity of interest $Q$, $\langle Q \rangle(R)$ -

\begin{equation}
    \langle Q \rangle(R) = \frac{\int_{-H(R)}^{H(R)} Q \, \rho \, \text{d}z} {\int_{-H(R)}^{H(R)} \rho \, \text{d}z} 
\end{equation}

where $H(R)$ is the scale height of the disk as a function of radius. This is also computed via a $z-$average as -

\begin{equation}
    H(R) =  \frac{\int_{\chi > 0.1} \rho \, H \, \text{d}z}{\int_{\chi > 0.1} \rho \, \text{d}z}
\end{equation}

here, $H = \sqrt{2}c_s/\Omega$ is the scale height based on the isothermal sound speed and the $z-$average is computed over the entire simulation domain. Despite the density-weighting, this averaging is noisy due to averaging over the fluff. To account for this, we impose the same cutoff between the torus material and the fluff as we use for refinement - i.e we only average over grid cells where the disk scalar $\chi > 0.1$, which is what the subscript over the integral signifies.

Shearing box results have shown that MRI turbulence has certain average magnetic properties that can be used as diagnostics. \citet{2011ApJ...738...84H} examined $\alpha_{\text{mag}} = M_{r\phi}/P_{\text{mag}} = -2B_RB_\phi/B^2$, the magnetic Shakura-Sunyaev alpha parameter which is the ratio of the Maxwell stress to the magnetic pressure as one such diagnostic. When suitably averaged over the computational domain, this quantity tends to approach specific values for well-resolved simulations.

While we're not examining the convergence of this quantity to the value that would indicate well-resolved MRI turbulence, we examine radial profiles of the averaged $\alpha_\text{mag}$ to examine the similarity of the AMR simulations to the corresponding uniform grid simulations. This is shown for the singleloop-beta100 simulation, averaged over $t = 200$ to  $t=300$ in Figure \ref{fig:radialalpha-single100}.

\begin{figure}
    \centering
    \includegraphics[width=0.7\linewidth]{figures/singleloop-beta100_mag-alpha_profile_200.0_300.0.pdf}
    \caption[Radial profile of the z-averaged $\alpha_\text{mag}$, averaged from $t=200$ to $t=300$ in the singleloop-beta100 simulation.]{Radial profile of the z-averaged $\alpha_\text{mag}$, averaged from $t=200$ to $t=300$ in the singleloop-beta100 simulation. Note the log scale, and the similar profiles for the AMR-lmax9 and UG-level9 profiles.}
    \label{fig:radialalpha-single100}
\end{figure}

The same plots for the singleloop-beta1000 and doubleloop-beta1000 simulations are shown in Figures \ref{fig:radialalpha-single1000} and \ref{fig:radialalpha-double1000}. The figures all show agreement between comparable AMR and uniform grid simulations, while highlighting differences between simulations with different refinement levels.

\begin{figure}
    \centering
    \includegraphics[width=0.7\linewidth]{figures/singleloop-beta1000_mag-alpha_profile_200.0_300.0.pdf}
    \caption[Radial profile of the z-averaged $\alpha_\text{mag}$, averaged from $t=200$ to $t=300$ in the singleloop-beta1000 simulation.]{Radial profile of the z-averaged $\alpha_\text{mag}$, averaged from $t=200$ to $t=300$ in the singleloop-beta1000 simulation. Note the stark drop-off in integrated $\alpha_\text{mag}$ at the outer edges of the torus in the lesser refined simulations.}
    \label{fig:radialalpha-single1000}
\end{figure}

\begin{figure}
    \centering
    \includegraphics[width=0.7\linewidth]{figures/doubleloop-beta1000_mag-alpha_profile_200.0_300.0.pdf}
    \caption[Radial profile of the z-averaged $\alpha_\text{mag}$, averaged from $t=200$ to $t=300$ in the doubleloop-beta1000 simulation.]{Radial profile of the z-averaged $\alpha_\text{mag}$, averaged from $t=200$ to $t=300$ in the doubleloop-beta1000 simulation. The agreement between comparable AMR and UG simulations and contrast among simulations with different refinement levels is more prominent at the inner edges of the torus.}
    \label{fig:radialalpha-double1000}
\end{figure}

\section{Computational Benefit}
\label{subsec:computation}

To quantitatively analyze the computational benefits afforded by utilizing Adaptive Mesh Refinement, we compare the number of cells in the computational domain as a function of time among comparable AMR and UG simulations for different simulations. The net benefit can be quantified by comparing the total number of cell advances among comparable simulations, defined as the integral of the total number of cells against timesteps.

These plots are shown for the singleloop-beta100 simulation in Figure \ref{fig:timesteps-singleloop100} with comparison among the comparable simulations of refinement level 8 (UG-level8 and AMR-lmax8) in \ref{subfig:timesteps-singleloop100-8} while comparison among the comparable simulations of refinement level 9 is shown in \ref{subfig:timesteps-singleloop100-9}.

\begin{figure}
    \centering
    \begin{subfigure}[t]{0.49\textwidth}
        \centering
        \includegraphics[width=0.9\textwidth]{figures/cellCount_singleloop-beta100_level8.pdf}
        \caption{Comparison among AMR and UG simulations with refinement level 8}
        \label{subfig:timesteps-singleloop100-8}
    \end{subfigure}%
    ~ 
    \begin{subfigure}[t]{0.49\textwidth}
        \centering
        \includegraphics[width=0.9\textwidth]{figures/cellCount_singleloop-beta100_level9.pdf}
        \caption{Comparison among AMR and UG simulations with refinement level 9}
        \label{subfig:timesteps-singleloop100-9}
    \end{subfigure}
    \caption{Number of cells against time for the singleloop-beta100 simulation}
    \label{fig:timesteps-singleloop100}
\end{figure}

The same figures for the singleloop-beta1000 and doubleloop-beta1000 simulations are shown in Figures \ref{fig:timesteps-singleloop1000} and \ref{fig:timesteps-doubleloop1000}. Overall, these highlight the factor $\sim 2 - 3$ decrease in number of cell advances required to achieve comparable simulation results (as seen via our metrics in \ref{subsec:global} and \ref{subsec:radial}) when utilizing AMR versus uniform grid simulations. We re-iterate here that our simulations are conducted in 2D axisymmetry. The computational benefits when utilizing a suitable refinement scheme in full 3D simulations will be even greater.

\begin{figure}
    \centering
    \begin{subfigure}[t]{0.49\textwidth}
        \centering
        \includegraphics[width=0.9\textwidth]{figures/cellCount_singleloop-beta1000_level8.pdf}
        \caption{Comparison among AMR and UG simulations with refinement level 8}
        \label{subfig:timesteps-singleloop1000-8}
    \end{subfigure}%
    ~ 
    \begin{subfigure}[t]{0.49\textwidth}
        \centering
        \includegraphics[width=0.9\textwidth]{figures/cellCount_singleloop-beta1000_level9.pdf}
        \caption{Comparison among AMR and UG simulations with refinement level 9}
        \label{subfig:timesteps-singleloop1000-9}
    \end{subfigure}
    \caption{Number of cells against time for the singleloop-beta1000 simulation}
    \label{fig:timesteps-singleloop1000}
\end{figure}

\begin{figure}
    \centering
    \begin{subfigure}[t]{0.49\textwidth}
        \centering
        \includegraphics[width=0.9\textwidth]{figures/cellCount_doubleloop-beta1000_level8.pdf}
        \caption{Comparison among AMR and UG simulations with refinement level 8}
        \label{subfig:timesteps-doubleloop1000-8}
    \end{subfigure}%
    ~ 
    \begin{subfigure}[t]{0.49\textwidth}
        \centering
        \includegraphics[width=0.9\textwidth]{figures/cellCount_doubleloop-beta1000_level9.pdf}
        \caption{Comparison among AMR and UG simulations with refinement level 9}
        \label{subfig:timesteps-doubleloop1000-9}
    \end{subfigure}
    \caption{Number of cells against time for the doubleloop-beta1000 simulation}
    \label{fig:timesteps-doubleloop1000}
\end{figure}

\chapter{Discussion and Conclusions}
\label{sec:discuss}

In this work, we have examined the feasibility of, and the computational benefit afforded by utilizing Adaptive Mesh Refinement (AMR) to adaptively refine simulations of accretion disks during the linear growth phase of the magnetorotational instability (MRI). The criterion we have used is \citet{2004ApJ...605..321S}'s resolution criteria for the resolution of MHD turbulence driven by the MRI, - $\langle \langle \lambda^2_\text{MRI}\rangle\rangle ^{1/2} \gtrsim 6 \Delta$, as generalized to the resolution of the linear MRI locally. This defines the well-studied quality factor metric for analyzing convergence of MHD simulations of disks.

Through the 2D axisymmetric simulations performed in this study, we find that AMR simulations utilizing this criterion show agreement in both space-integrated global quantities like the total mass $M$ and the poloidal magnetic energy $B_\text{pol}$, and spatially varying time and z-averaged radial quantities (in particular $\alpha_\text{mag}$). Because AMR simulations refine the computational domain selectively, the total number of cells is lower for the AMR simulations and we observe a factor $\sim 2-3$ reduction in the total number of cell advances for the AMR simulations as compared to the comparable uniform grid simulations.

Cowling's antidynamo theorem \citep{1933MNRAS..94...39C} means that our simulations cannot sustain a poloidal magnetic field owing to the two-dimensional nature of our simulations. Analyzing the applicability of this resolution criteria to 3D non-axisymmetric simulations is one of the avenues for future work. In a recent study, \citet{jannaud_numerical_2025} analyzed the issues with utilizing the quality factor as a metric of convergence and as a resolution criteria, highlighting issues particularly in the case of a zero net magnetic flux. They also highlight issues with the assumptions underpinning the quality factor - a linear theory assuming a constant net magnetic flux and thus, a well-defined minimum MRI lengthscale. They instead propose a revised quality factor, which is a function of the net magnetic flux threading the disk and the quality factor associated with said flux.

Our work offers a contrast to that study, showcasing the applicability of the quality factor discussed in literature to at least 2D axisymmetric simulations of accretion disks during the linear growth phase of the poloidal MRI. At the very least, we hope to inspire future work utilizing Adaptive Mesh Refinement (AMR) as a tool to better utilize available computing resources in MHD simulations of disks.


%%  Everything after this is an appendix
%%  ------------------------------------

%%%
%%%  If there is just ONE appendix use the \singleappendix command
%%%
%\singleappendix
%\chapter{The Only Appendix}
%
%If there is only one appendix, it is called ``Appendix'' (not ``Appendix
%A'').  
%
%To achieve this, use the command, {\tt $\backslash$singleappendix}
%
%\section{Appendictical Numbering}
%\label{sample-appendix:numbering-section}
%
%The command, {\tt $\backslash$singleappendix} prints the appendix title without
%the trailing A.
%
%\section{Getting the labels right \ldots}
%
%The {\tt $\backslash$singleappendix} command also ensures that the section and
%subsection numbers don't include a leading A. and that the the Table of
%Contents, List of Figures and List of Tables entries for section, subsection,
%equations, figures and tables don't have a leading period. 
%

%%
%%  If there is more that ONE appendix
%%
\appendix

% \chapter{Additional Material
% \label{sample-appendix}}

% \chapter{The Second Appendix}



%%
%%  End of appendices
%%  =========================================================================


%%  At the end of the document are the references. These are single-spaced
%%  rather than double-spaced like the rest of the thesis text.

\begin{singlespace}
    \bibliography{physicsMasters}
\end{singlespace}


\end{document}
%% ===========
%%    FINI
%% ===========